
\documentclass[12pt]{article}
\usepackage{amsthm}
\usepackage{fullpage}
\usepackage{epsfig}
\usepackage{amsmath}
\usepackage{amsfonts}
\usepackage[normalem]{ulem}
\usepackage{color}
\usepackage{amssymb, clrscode3e}


\newtheorem{theorem}{Theorem}[section]
\newtheorem{corollary}{Corollary}[theorem]
\newtheorem{lemma}[theorem]{Lemma}
\newtheorem{claim}{Claim}

\usepackage{parskip}

\begin{document}

Units: 
\begin{enumerate}
    \item Floating points
\item Interpolation
\item Ordinary Differential Equations
\item Fourier Analysis
\item Numerical Linear Algebra

\end{enumerate}

Grades: Assignment: 32\%, 4\%, Midterm: 28\%, Exam: 40\%

Midterm at June 26, 7pm

Printed course notes at Media doc in DC. 


\section{Floating Point Arithmetic}

Real numbers are: 
\begin{itemize}
\item infinite in extent
\item infinite in density
\end{itemize}

Floating point system is an approximate representation of real numbers using a finite number of bits. 

Consider the sum
$$12 + \sum^{100}_{i=1} 0.01 = 13$$

If we perform the sum one addition at a time, retaining two digts of accuracy at each step: 
$$12 + 0.01 + 0.01 + \dots $$

The numerical answer is 12. Performing the addition from the opposite direction will yield the correct answer. 

We can express a real number as an infinite expansion relative to some base $\beta$. After expressing the real number in the desired base $\beta$, we multiply it by a power of $\beta$ to shift it into a normalized form

$$0.d_1d_2d_3d_4 \dots \times \beta^\rho$$

where
\begin{itemize}
\item $d_i$ are digits in base $B$, i.e., $0 \leq d_i \le \beta$
\item normalized means $d_1 \neq 0$
\item exponent $\rho$ is an integer
\end{itemize}

\subsection{Floating Point System}

Density (or precision) is bounded by the number of digits, $t$  
Extent (or range) is bounded by limiting the range of values for $\rho$. 

Our floating point representation then has the form: 
$$  
\left\{\begin{aligned}  
&\pm 0.d_1 d_2 \dots d_t \times \beta^\rho &&: L \leq \rho \leq U, d_1 \neq 0\\  
&0 
\end{aligned}  
\right.$$

The four integer parameters $\{\beta, t, L, U\}$ characterize a specific floating point system, $F$. 


If $\rho > U$ or $\rho < L$, our system cannot represent the number. When an arithmetic operation generates such a number, it's called overflow or underflow

The floating point standards are almost always directly implemented in the hardware. The two most common standardized floating point systems are: 
\begin{itemize}
    \item IEEE single precision (32 bits): $\beta = 2, t=24, L=-126, U=127$
    \item IEEE double precision (64 bits): $\beta = 2, t=53, L=-1022, U=1033$
\end{itemize}

\begin{itemize}
    \item Fixed number of digits are the decimal; integer representation scaled by a fixed scale factor; eg. $10234 \times 10^{-3}$
    \item Floating point number systems let the radix point float to represent a wider range.  
    \item Floating point numbers are not evenly spaced
\end{itemize}


\subsection{Measuring Errors}

Absolute error: 
$$E_{abs} = | x_{exact} - x_{approx} |$$

Relative error: 
$$E_{rel} = \frac{|x_{exact}| - |x_{approx}|} {|x_{exact}|}$$


Relative error is more useful: 
\begin{itemize}
\item independent of magnitudes of numbers involved
\item related to the number of significant digits in the result
\end{itemize}

A result is correct to approximately $s$ digits if $E_{rel} \approx 10^{-s}$ or $$0.5 \times 10^{-s} \leq E_{req} \le 5 \times 10^{-s}$$

For any given FP system $F$, there exists an upper bound $E$ such that 
$$(1-E)|x| \leq |fl(x) | \leq (1+E) |x|$$. 

Machine epsilon / unit round-off error: The maximum relative error $E$ for a FP system. Small value such that $fl (1+E) > 1$ under $F$. 

In an FP system $F(\beta, t, Lu, U)$ with rounding to the nearest, $$E = \frac{1}{2} \beta^{1-t}$$

With truncation, 
$$E = \beta^{1-t}$$

Thus, we have the signed relative error $\delta = (fl(x) - x) / x$, or $$fl(x) = x (1+\delta)$$ with $|\delta| \leq E$. 

For practical purpose, we only care about the order of magnitude of $E$ 



\end{document}

